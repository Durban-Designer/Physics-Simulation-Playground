\subsection{Disorder Engineering (Literature-Based)}
Experimental studies on materials like cuprates have shown that $T_c$ can be enhanced by introducing a specific, optimal amount of disorder. This effect is highly material-dependent and non-linear. Instead of using arbitrary linear models, our approach is to model this phenomenon by fitting to experimental data from the literature. A proper model for disorder enhancement would take the form of a function fitted to known experimental curves:
\begin{lstlisting}[language=Python, caption={Proposed model for disorder enhancement}]
def disorder_enhancement(material, disorder_strength, defect_type):
    """Based on experimental literature for specific materials"""

    # Material-specific optimal disorder from experiments
    optimal_disorder = material.experimental_optimal_disorder

    # Fitted to experimental enhancement curves
    if disorder_strength < optimal_disorder:
        enhancement = 1 + material.enhancement_slope * disorder_strength
    else:
        # Anderson localization suppression
        suppression = np.exp(-(disorder_strength - optimal_disorder) / material.localization_length)
        enhancement = material.max_enhancement * suppression

    return enhancement
\end{lstlisting}
This data-driven approach ensures that the model's predictions are grounded in physical reality.

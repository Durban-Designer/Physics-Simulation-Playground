\subsection{Uncertainty Quantification}
Predictions from a complex model are incomplete without a quantification of their uncertainty. We will employ Bayesian parameter estimation to achieve this. Instead of point estimates, our model will return a probability distribution for the predicted $T_c$.
\begin{lstlisting}[language=Python, caption={Bayesian Tc Prediction Workflow}]
def bayesian_tc_prediction(material_params, disorder, experimental_data):
    """
    Returns: (tc_mean, tc_std, confidence_interval)

    Uses Markov Chain Monte Carlo to sample parameter posterior
    given experimental constraints.
    """

    # Prior distributions from literature
    priors = get_literature_priors(material_params)

    # Likelihood from experimental data
    likelihood = experimental_likelihood(experimental_data)

    # MCMC sampling
    posterior_samples = mcmc_sample(priors, likelihood)

    # Propagate uncertainty through physics model
    tc_distribution = [physics_model(sample) for sample in posterior_samples]

    return statistics(tc_distribution)
\end{lstlisting}
This approach allows us to incorporate uncertainties from both experimental inputs and the model itself, providing a much more honest and useful prediction.
